\documentclass[10pt]{article}

% Core math & graphics
\usepackage{amsmath}
\usepackage{graphicx}
\usepackage[dvipsnames]{xcolor} % load once, with options

% Floats, tables, layout
\usepackage{wrapfig}
\usepackage{tabularx}
\usepackage{array}
\usepackage{booktabs}

% Units/symbol helpers (see notes about gensymb below)
\usepackage{gensymb} % OK; if conflicts with \micro or \perthousand, consider siunitx

% Page geometry — choose margins OR total area (here: margins)
\usepackage[
  top=20mm,
  bottom=25mm,
  left=15mm,
  right=15mm
]{geometry}

% Captions/subcaptions (caption options applied before subcaption)
\usepackage[labelfont=bf]{caption}
\usepackage{subcaption}

% Misc utilities
\usepackage[export]{adjustbox}
\usepackage{setspace}
\usepackage{listings} % load once
\usepackage{mdframed}
\usepackage{tikz}
\usepackage{url}
\usepackage{marginnote}
\usepackage{csquotes}
\usepackage[normalem]{ulem} % keep \emph as italics
\usepackage{multicol}
\usepackage{lipsum}
\setlength{\parindent}{20pt}

% Number equations by section
\numberwithin{equation}{section}

% Nomenclature
\usepackage[intoc]{nomencl}
\makenomenclature

% Hyperlinks (keep near the end)
\usepackage{hyperref}
\hypersetup{
  colorlinks,
  citecolor=black,
  filecolor=black,
  linkcolor=black,
  urlcolor=black
}




\begin{document}


\begin{flushleft}

\onehalfspacing
\tableofcontents
\newpage

\nomenclature{$\mathcal{N}$}{Total number of moles in the system}
\nomenclature{$\mathcal{N}_j$}{Number of moles of species $j$ in the system}
\nomenclature{$\mathcal{L}$}{The Lagrangian}
\nomenclature{$a_{ij}$}{Stoichiometric coefficients, e.g. number of atoms of element $i$ in species $j$}
\nomenclature{$\hat{R}$}{Universal gas constant $\approx$ 8.314 J/kg-K}
\nomenclature{$b_i$}{Sum of contributions from each species towards constraint for element $i$, i.g. $\sum_{j=1}^\text{NS} a_{i,j}\mathcal{N}_j$}
\nomenclature{$p_\text{ref}$}{Reference temperature used to compute chemical potential}
\nomenclature{$G$}{Gibbs free-energy}
\nomenclature{$F$}{Helmholtz energy}
\nomenclature{$T$}{Temperature (K)}
\nomenclature{$q_j$}{Charge of species $j$ (0 for neutrals, 1 for ions, -1 for electrons)}
\nomenclature{$V$}{Volume ($m^3$)}
\nomenclature{$\rho$}{Density ($kg/m^3$}
\nomenclature{$MW_j$}{Molecular weight of species $j$}
\nomenclature{$\lambda$}{Lagrange multiplier}
\nomenclature{$\mu_j$}{Chemical potential of species $j$}
\nomenclature{$\mu_j^\circ$}{Standard state chemical potential of species $j$}
\nomenclature{$b_i^\circ$}{Specified number of moles of element $i$}
\nomenclature{$U_j^\circ$}{Standard state internal energy of species $j$}
\nomenclature{$H_j^\circ$}{Standard state enthalpy of species $j$}.
\nomenclature{$c_{v,j}^\circ$}{Standard state specific heat of species $j$}
\nomenclature{$c_{p,j}^\circ$}{Standard state specific heat of species $j$}
\nomenclature{$u_0'$}{User-specified internal energy}
\nomenclature{$h_0'$}{User-specified enthalpy}
\nomenclature{$u'$}{Sum of contributions from each species towards internal energy constraint, i.g. $\sum_{j}^\text{NS} \mathcal{N}_j U_j^\circ$}
\nomenclature{$h'$}{Sum of contributions from each species towards enthalpy  constraint, i.g. $\sum_{j}^\text{NS} \mathcal{N}_j H_j^\circ$}


\begin{multicols}{2}
\printnomenclature    
\end{multicols}


\section{Minimization Procedures}

\subsection{Constraints for Both Procedures}

There are two constraints that are always used in the minimization procedures. The first is the elemental constraint:

\begin{equation}
    b_i^\circ - \sum_{j=1}^\text{NS} a_{ij} \mathcal{N}_j = 0
    \label{elementalconstraint}
\end{equation}

The second is the charge constraint:
\begin{equation}
    \sum_{j=1}^\text{NS}q_j \mathcal{N}_j = 0
    \label{chargeconstraint}
\end{equation}

Since these are used in both procedures, that are put in the Lagrangian term. Other constraints (such specified volume, temperature, internal energy, and entropy) are given their own equations to satisfy their own conditions.

\subsection{Newton-Raphson Iterative Method}

We utilize a Newton-Raphson method to minimize the problem. Since this is essentially a root finding system of equations, and we are seeking the solution update, the method looks like the following:

\begin{equation}
    \sum_{i=1}^N \frac{\partial f}{\partial x_i} \Delta x_i = -f(x)
    \label{newton}
\end{equation}

Here, $f(x)$ is the function that is being minimized. In the case of Gibbs and Helmholtz energy minimization, $f$ takes the form of $\partial \mathcal{L} / \partial \eta$, with $\eta$ being $\mathcal{N}_j$, $\pi_i$, or $\pi_q$. We use non-linear correction variables of the form $\Delta \ln \mathcal{N}_j$, $\Delta \ln\mathcal{N}$, $\Delta \ln T$, and $\pi_{i,q} = \Delta \pi_{i,q}$. For the last correction variable, it is argued that setting $\pi$ to 0 at the beginning of each update does not influence the solution, so we drop the $\Delta$ for that term. This will influence the form of the update equations, which will be talked about in due time.



\subsection{Gibbs Minimization}

\begin{equation}
    G = G(T, p, \mathcal{N}_1, \mathcal{N}_2, ... , \mathcal{N}_\text{NS}) = \sum_{j=1}^\text{NS}\mathcal{N}_j \mu_j
    \label{gibbs}
\end{equation}

Where NS is the number of gas species, and $\mathcal{N_j}$ is the number of moles of species $j$, and $\mu_j$ is the chemical potential of species $j$ defined by:

\begin{equation}
    \mu_j = \frac{\partial G}{\partial \mathcal{N}_j} = \mu_j^\circ + \hat{R}T \left[\ln\left(\frac{p}{p_\text{ref}}\right) + \ln \left( \frac{\mathcal{N}_j}{\mathcal{N}} \right) \right]
    \label{chempot}
\end{equation}

Here, $\hat{R}$ is the universal gas constant, $T$ and $p$ are the temperature and pressure of the system, $p_\text{ref}$ is the reference pressure (usually takes as 101,325 Pa, or 1 bar depending on the literature), and $\mathcal{N}$ is the total number of moles in the system. 

In order to find chemical equilibrium the derivative of $G$ wrt $\mathcal{N}_j$ is set to 0 for all species. A number of constraints need to be added in. Firstly, the elemental constraint:

\begin{equation}
    \sum_{j=1}^\text{NS}a_{ij} \mathcal{N}_j - b_i^\circ= 0
    \label{elemental}
\end{equation}

Equation $\ref{elemental}$ says that the number of moles of element $i$ must remain constant during the minimization process. $b_i$ is the number of moles of element $i$, and $a_{i,j}$ is the stoichiometric coefficient, e.g. how many atoms of element $i$ are in species $j$. A secondary constraint equation can be used:

\begin{equation}
    \mathcal{N} - \sum_{j=1}^\text{NS}\mathcal{N}_j = 0
\end{equation}

This ensures that the total moles in the system is equal to the sum of the moles of each individual species. Finally, a tertiary constraint ensures that the charge of the system remains neutral:

\begin{equation}
    \sum_{j = 0}^\text{NS}q_j\mathcal{N}_j = 0
\end{equation}


Minimization is achieved through use of the Lagrangian, $\mathcal{L}$. The Lagrangian is defined as

\begin{equation}
    \mathcal{L} = f + \sum_{i} \lambda_i g_i
\end{equation}

Where $f$ is our original function (equation \ref{gibbs}), and $g_i$ are our constraint equations. Therefore, the total Lagrangian is then:

\begin{equation}
    \mathcal{L} = \sum_{j=1}^\text{NS} \mu_j\mathcal{N}_j + \sum_{i=1}^\text{NE}\lambda_i 
    \left[\sum_{j=1}^\text{NS}a_{ij} \mathcal{N}_j - b_i^\circ\right] + \lambda_q \left[ \sum_{j = 0}^\text{NS}q_j\mathcal{N}_j - 0 \right]
\end{equation}

We now take the derivatives of $\mathcal{L}$ wrt $\mathcal{N}_j$, and each $\lambda$:

\begin{align}
    \frac{\partial \mathcal{L}}{\partial \mathcal{N}_j} &= f_1 = \mu_j + \sum_{i=1}^\text{NE}\lambda_i a_{ij} + \lambda_q q_j = 0
    \label{f1}
    \\ \frac{\partial \mathcal{L}}{\partial \lambda_i} &= f_2 =  \sum_{i=1}^\text{NS} a_{ij} \mathcal{N}_j - b_i = 0
    \label{f2}
    \\ \frac{\partial \mathcal{L}}{\partial \lambda_q} &= f_3 = \sum_{j=0}^\text{NS}q_j\mathcal{N}_j - 0 = 0
    \label{f3}
\end{align}

These three equations are our minimization functions. Through use of a Newton-Raphson method, we can solve these equations. First, we must do some algebraic manipulation. We start by expanding equation \ref{f1} with equation \ref{chempot}:

\begin{equation}
    \frac{\partial \mathcal{L}}{\partial \mathcal{N}_j} = \mu_j^\circ + \hat{R}T \left[\ln\left(\frac{p}{p_\text{ref}}\right) + \ln \left( \frac{\mathcal{N}_j}{\mathcal{N}} \right) \right] + \sum_{i=1}^\text{NE}\lambda_i a_{ij} + \lambda_q q_j = 0
\end{equation}

Dividing through by $\hat{R}T$ and setting $\pi = -\lambda/\hat{R}T$ gives:

\begin{equation}
    \frac{\partial \mathcal{L}}{\partial \mathcal{N}_j} = f_1 = \frac{\mu_j^\circ}{\hat{R}T} + \ln{\frac{p}{p_\text{ref}}} + ln{\mathcal{N}_j} - \ln{\mathcal{N}} - \sum_{i=1}^\text{NE}\pi_ia_{i,j} - \pi_q q_j = 0
\end{equation}

We introduce linear correction variables $\Delta \ln{\mathcal{N}_j}$ and $\Delta \ln{\mathcal{N}}$. In order to use the Newton method, we need to solve the equation:

Here, $f(x)$ are equations \ref{f1} - \ref{f3}. We now need to take the partial derivatives of our function wrt our non-linear variables as well as each $\pi$, and then multiple it by the correction variables. We start with equation \ref{f1}:

\begin{align}
    \frac{\partial f_1}{\partial[\ln{\mathcal{N}_j}]} \Delta\ln{\mathcal{N}_j} &= \Delta\ln{\mathcal{N}_j}
    \\ \frac{\partial f_1}{\partial[\ln{\mathcal{N}}]} \Delta\ln{\mathcal{N}} &= -\Delta\ln{\mathcal{N}}
    \\ \frac{\partial f_1}{\partial [\pi_i]} \pi_i &= -a_{ij} \pi_i
    \\ \frac{\partial f_1}{\partial [\pi_q]} \pi_q &= -q_j \pi_q
\end{align}

Combining these into the form of equation \ref{newton} gives

\begin{equation}
    \Delta\ln{\mathcal{N}_j} - \Delta \ln{\mathcal{N}} - \sum_{i=1}^\text{NE}-a_{ij}\pi_i = \mu_j + \sum_{i=1}^\text{NE}\lambda_i a_{ij} + \lambda_q q_j
\end{equation}

Arguments are made that $\pi_i$ = 0 at the start of every iteration, so this equation becomes:


\begin{equation}
    \Delta\ln{\mathcal{N}_j} - \Delta \ln{\mathcal{N}} - \sum_{i=1}^\text{NE}-a_{ij}\pi_i - q_j \pi_q= -\frac{\mu_j}{\hat{R}T}
\end{equation}

There will be one of these equation for each species (NS). 


\subsection{Helmholtz Minimization}

The Helmholtz energy is defined as:

\begin{equation}
    F = G - pV
\end{equation}

Where $G$ is the Gibbs free energy, $p$ is the pressure, and $V$ is the volume. Substitution of $G$ yields:

\begin{equation}
    F = \sum_{j=1}^\text{NS} \mu_j \mathcal{N}_j - pV
\end{equation}

Where the chemical potential is redefined as:

\begin{equation}
    \mu_j = \mu_j^\circ + \hat{R}T \ln\left( \frac{\mathcal{N}_j R'T}{V} \right)
    \label{chempotF}
\end{equation}

and $R' = \hat{R} \cdot 10^{-5}$. This is subject to the elemental constraint, and the ionic constraint if ions are present. For assigned temperature and volume, these are the only constraints you need. If you want to assign internal energy and volume, you constrain as such

\begin{equation}
    u' = \sum_{j=1}^\text{NS} \mathcal{N}_j U_j^\circ = u_0'
\end{equation}

Which yields:

\begin{equation}
    u' - u_0' = 0
\end{equation}

Where $u_0'$ is the specified internal energy. While you may be able to include the constraints for internal energy and entropy inside the Lagrangian, we will exclude them from the Lagrangian and use our own residual equation for them. The Lagrangian for the Helmholtz energy minimization process is then defined as:

\begin{equation}
    \mathcal{L} = \sum_{j=1}^\text{NS} \mu_j \mathcal{N}_j + \sum_{i=1}^\text{NE}\lambda_i \left[ \sum_{j=1}^\text{NS}a_{ij} \mathcal{N}_j - b_i^\circ \right] + \lambda_q \left[\sum_{j=1}^\text{NS} q_j \mathcal{N}_j - 0 \right]
    \label{helmholtzL}
\end{equation}

The first order of business is to take derivatives wrt $\mathcal{N}_j,$ $\lambda_i$, and $\lambda_q$:

\begin{align}
    \frac{\partial \mathcal{L}}{\partial \mathcal{N}_j} &= f_1 = \mu_j + \sum_{i=1}^\text{NE}\lambda_ia_{ij} + \lambda_q q_j = 0
    \\ \frac{\partial \mathcal{L}}{\partial \lambda_i} &= f_2 = \sum_{j=1}^\text{NS} a_{ij} \mathcal{N}_j - b_i^\circ= 0
    \\ \frac{\partial \mathcal{L}}{\partial \lambda_q} &= f_3 = \sum_{j=1}^\text{NS} q_j \mathcal{N}_j - 0= 0
\end{align}

Then, we work on the second derivatives that define our system of equations. First, we expand the chemical potential term using equation \ref{chempotF}. WE then non-dimensionalize the equation by $\hat{R}T$. Finally, we make sure to convert these equations into a form that can be differentiated by our non-linear variables. We also set $\pi = -\lambda/\hat{R}T$. For Helmholtz equations, the non-linear variables are $\ln\mathcal{N}_j$, $\ln T$, and $\pi_{i,q}$.

\begin{align}
    f_1(\ln\mathcal{N}_j) &= \frac{\mu_j^\circ}{\hat{R}T} + \ln \mathcal{N}_j + \ln\frac{R'}{V} + \ln T - \sum_{i=1}^\text{NE}\pi_i a_{ij} - \pi_q q_j
    \\ f_2(\ln \mathcal{N}_j) &= \sum_{j=1}^\text{NS} a_{ij} \exp(\ln\mathcal{N}_j) - b_i^\circ 
    \\ f_3(\ln \mathcal{N}_j) &= \sum_{j=1}^\text{NS} q_j \exp(\ln\mathcal{N}_j) - 0
\end{align}

We know go through and take the derivatives and multiply them by the correction variables. For $f_1$, we get:

\begin{align}
    \frac{\partial f_1}{\partial [\ln \mathcal{N}_j]} \Delta \ln \mathcal{N}_j &= \Delta \ln \mathcal{N}_j
    \\ \frac{\partial f_1}{\partial [\ln T]} \Delta \ln T &= -\frac{U_j^\circ}{\hat{R}T} \ln T
    \label{f1lnT}
    \\ \frac{\partial f_1}{\partial \pi_i} \pi_i &= -a_{ij} \pi_i
    \\ \frac{\partial f_1}{\partial \pi_q} \pi_q &= -q_j \pi_q
\end{align} 

The derivation for equation [\ref{f1lnT}] is carried out in more depth in section [\ref{derivativesF}]. Therefore, our final result is:

\begin{equation*}
    \boxed{\Delta \ln \mathcal{N}_j - \sum_{i=1}^\text{NE}a_{ij} \pi_i - q_j \pi_q -\frac{U_j^\circ}{\hat{R}T} \Delta \ln T  = -\frac{\mu_j}{\hat{R}T} }
\end{equation*}

There will be NS number of these equations in our system. Next, we take the derivatives of $f_2$:

\begin{align}
    \frac{\partial f_2}{\partial [\ln \mathcal{N}_j]} \Delta \ln \mathcal{N}_j &= \mathcal{N}_j a_{i,j} \Delta \ln \mathcal{N}_j
    \\ \frac{\partial f_2}{\partial [\ln T]} \Delta \ln T &= 0
    \\ \frac{\partial f_2}{\partial \pi_i} \pi_i &= 0
    \\ \frac{\partial f_2}{\partial \pi_q} \pi_q &= 0
\end{align} 

Which gives:

\begin{equation*}
    \boxed{ \sum_{j=1}^\text{NS}\mathcal{N}_j a_{ij} \Delta \ln \mathcal{N}_j = b_i^\circ - \sum_{j=1}^\text{NS} a_{ij} \mathcal{N}_j}
\end{equation*}

There will be NE of these equations. Finally, for $f_3$ we get:

\begin{align}
    \frac{\partial f_3}{\partial [\ln \mathcal{N}_j]} \Delta \ln \mathcal{N}_j &= \mathcal{N}_j q_j \Delta \ln \mathcal{N}_j
    \\ \frac{\partial f_3}{\partial [\ln T]} \Delta \ln T &= 0
    \\ \frac{\partial f_3}{\partial \pi_i} \pi_i &= 0
    \\ \frac{\partial f_3}{\partial \pi_q} \pi_q &= 0
\end{align} 

Which gives only one equation:

\begin{equation*}
    \boxed{ \sum_{j=1}^\text{NS}\mathcal{N}_j q_j \Delta \ln \mathcal{N}_j = -\sum_{j=1}^\text{NS} q_j \mathcal{N}_j}
\end{equation*}

Finally, we look towards the constraints, mainly setting internal energy.

\begin{equation}
    f_4(\ln \mathcal{N}_j, \ln T) = \sum_{j=1}^\text{NS} \mathcal{N}_j U_j^\circ - u_0' = u' - u_0'
\end{equation}

Using identity [\ref{dTID}] helps with the temperature derivative. The derivatives are then:

\begin{align}
    \frac{\partial f_4}{\partial [\ln \mathcal{N}_j]} \Delta \ln \mathcal{N}_j &= \frac{\mathcal{N}_jU_j^\circ}{\hat{R}T} \Delta \ln \mathcal{N}_j
    \\ \frac{\partial f_4}{\partial [\ln T]} \Delta \ln T &= \sum_{j=1}^\text{NS} [T \mathcal{N}_jc_{v,j}^\circ] \Delta \ln T
    \\ \frac{\partial f_4}{\partial \pi_i} \pi_i &= 0
    \\ \frac{\partial f_4}{\partial \pi_q} \pi_q &= 0
\end{align}

After non-dimensionalizing, we are left with:

\begin{equation*}
    \boxed{ \sum_{j=1}^\text{NS}\frac{\mathcal{N}_jU_j^\circ}{\hat{R}T} \Delta \ln \mathcal{N}_j + \sum_{j=1}^\text{NS} \frac{ \mathcal{N}_jc_{v,j}^\circ}{R} \Delta \ln T = \frac{u_0' - u'}{\hat{RT}}}
\end{equation*}


For convenience, they are all listed here together:

\begin{align}
        &\Delta \ln \mathcal{N}_j - \sum_{i=1}^\text{NE}a_{ij} \pi_i - q_j \pi_q -\frac{U_j^\circ}{\hat{R}T} \Delta \ln T  = -\frac{\mu_j}{\hat{R}T} 
        \label{HdlnNj}
        \\ &\sum_{j=1}^\text{NS}\mathcal{N}_j a_{ij} \Delta \ln \mathcal{N}_j = b_i^\circ - \sum_{j=1}^\text{NS} a_{ij} \mathcal{N}_j
        \label{Hdpi}
        \\ &\sum_{j=1}^\text{NS}\mathcal{N}_j q_j \Delta \ln \mathcal{N}_j = -\sum_{j=1}^\text{NS} q_j \mathcal{N}_j
        \label{Hdpiq}
        \\ &\sum_{j=1}^\text{NS}\frac{\mathcal{N}_jU_j^\circ}{\hat{R}T} \Delta \ln \mathcal{N}_j + \sum_{j=1}^\text{NS} \frac{ \mathcal{N}_jc_{v,j}^\circ}{R} \Delta \ln T = \frac{u_0' - u'}{\hat{RT}}
        \label{HdlnT}
\end{align}

\subsubsection{Reduced Helmholtz Equations}

A "simplification" can be made to the above system of equations by solving equations [\ref{HdlnNj}] for $\Delta \ln \mathcal{N}_j$:

\begin{equation}
    \textcolor{red}{\Delta \ln \mathcal{N}_j =  \sum_{i=1}^\text{NE}a_{i,j} -\frac{\mu_j}{\hat{R}T} \pi_i} \textcolor{blue}{+ q_j \pi_q}\textcolor{orange}{ + \frac{U_j^\circ}{\hat{R}T} \Delta \ln T }
    \label{redH0}
\end{equation}

By substituting this into equations [\ref{Hdpi}] - [\ref{HdlnT}], we can shrink the system of equations dramatically. The resulting set of equations is:

\begin{equation}
    \textcolor{red}{\sum_{i=1}^\text{NE} \left[ \sum_{j=1}^\text{NS} a_{kj} a_{ij} \mathcal{N}_j \right] \pi_i} \textcolor{blue}{+ \left[\sum_{j=1}^\text{NS} a_{kj} q_j\mathcal{N}_j \right] \pi_q} \textcolor{orange}{+ \left[ \sum_{j=1}^\text{NS} a_{kj} \mathcal{N}_j\frac{ U_j^\circ}{\hat{R}T} \right] \Delta \ln T} = \textcolor{red}{b_k^\circ - \sum_{j=1}^\text{NS} a_{kj} \mathcal{N}_j +  \sum_{j=1}^\text{NS} a_{kj} \mathcal{N}_j\frac{ \mu_j}{\hat{R}T}}
    \label{redH1}
\end{equation}

\begin{equation}
    \textcolor{blue}{\sum_{i=1}^\text{NE} \left[ \sum_{j=1}^\text{NS} a_{ij} q_j \mathcal{N}_j \right] \pi_i + \left[ \sum_{j=1}^\text{NS}  q_j^2 \mathcal{N}_j \right] \pi_q} \textcolor{orange}{+ \left[ \sum_{j=1}^\text{NS} q_j \mathcal{N}_j\frac{ U_j^\circ}{\hat{R}T} \right] \Delta \ln T} = \textcolor{blue}{\sum_{j=1}^\text{NS} q_j \mathcal{N}_j \frac{\mu_j}{\hat{R}T} - \sum_{j=1}^\text{NS} q_j \mathcal{N}_j}
    \label{redH2}
\end{equation}

\begin{equation}
    \textcolor{orange}{\sum_{i=1}^\text{NE} \left[ \sum_{j=1}^\text{NS} a_{ij} \mathcal{N}_j\frac{ U_j^\circ}{\hat{R}T} \right] \pi_i} \textcolor{blue}{+ \left[ \sum_{j=1}^\text{NS} q_j \mathcal{N}_j \frac{ U_j^\circ}{\hat{R}T} \right] \pi_q} \textcolor{orange}{+ \left[ \sum_{j=1}^\text{NS} \mathcal{N}_j \left( \frac{ U_j^{\circ}}{\hat{R} T} \right)^2 + \sum_{j=1}^\text{NS} \mathcal{N}_j \frac{ c_{v,j}^\circ}{\hat{R}} \right] \Delta \ln T = \frac{u_0' - u'}{\hat{R}T} + \sum_{j=1}^\text{NS} \mathcal{N}_j \frac{ U_j^\circ}{\hat{R} T} \frac{\mu_j}{\hat{R}T}}
    \label{redH3}
\end{equation}


There will be NE number of equations $\ref{redH1}$, and one of both $\ref{redH2}$ and $\ref{redH3}$. 

\subsection{Useful Relationships}

\begin{equation}
    \frac{c_v^\circ}{\hat{R}} = \frac{c_{p,j}^\circ}{\hat{R}} -1
\end{equation}

\begin{equation}
    \frac{\mu_j^\circ}{\hat{R}T} = \frac{H_j^\circ}{\hat{R}T} - \frac{S_j^\circ}{\hat{R}} 
\end{equation}

\begin{equation}
    \frac{U_j^\circ}{\hat{R}T} = \frac{H_j^\circ}{\hat{R}T} - 1
\end{equation}

\section{Standard State Chemical Potential Derivatives}

Some important derivatives are derived here. First, we state a chain rule identity that is very useful:

\begin{equation}
    \frac{\partial}{\partial T} = \frac{\partial [\ln (T)]}{\partial T} \frac{\partial}{\partial [\ln(T)]} \rightarrow \frac{\partial}{\partial [\ln(T)]} = T \frac{\partial}{\partial T}
    \label{dTID}
\end{equation}

Then, taking the derivative of $\mu_j^0$ gives:

\begin{equation}
    \frac{\partial}{\partial [\ln(T)]} \left( \frac{\mu_j^0}{RT} \right)  = T \frac{\partial }{\partial T} \left( \frac{\mu_j^0}{RT} \right)
\end{equation}

We expand $\mu_j^0 /RT$ as:

\begin{align}
    \frac{\mu_j^0}{RT} &= \frac{H_j^0}{RT} - \frac{S_j^0}{R} 
    \label{mu0_H}
    \\ \frac{\mu_j^0}{RT} &= \frac{U_j^0}{RT} - \frac{S_j^0}{R} + 1
    \label{mu0_U}
\end{align}

\subsection{Gibbs Derivative}
\label{derivativesG}
First, we take the derivative of [\ref{mu0_H}] for Gibbs minimization.

\begin{equation}
    T \frac{\partial }{\partial T} \left( \frac{H_j^0}{RT} - \frac{S_j^0}{R} \right) = T \left( \frac{\partial_T(H_j^0) \cdot RT - H_j^0\cdot R}{(RT)^2} - \frac{\partial_T(S_j^0)}{R}\right)
     \label{dmu0_H}
\end{equation}

Here, $\partial_T$ denotes ($\partial/ \partial T$). If:

\begin{equation}
    c_{p,j}^0 = \frac{\partial H_j^0}{\partial T}, \hspace{2mm} \text{and} \hspace{2mm} \frac{c_{p,j}^0}{T} = \frac{\partial S_j^0}{\partial T} 
    \label{dSdT}
\end{equation}

This reduces to:

\begin{equation}
    T \left [  \frac{c_{p,j}^0}{RT} - \frac{H_j^0}{RT^2}  - \frac{c_{p,j}^0}{RT} \right] = -\frac{H_j^0}{RT}
\end{equation}

Therefore:

\begin{equation}
    \boxed{\frac{\partial}{\partial[\ln(T)]} \left( \frac{\mu_j^0}{RT} \right) = -\frac{H_j^0}{RT} }
\end{equation}

\subsection{Helmholtz Derivative}
\label{derivativesF}

Using equation [\ref{mu0_U}] gives us:

\begin{equation}
    T \frac{\partial }{\partial T} \left( \frac{U_j^0}{RT}  + 1- \frac{S_j^0}{R} \right) = T \left( \frac{\partial_T(U_j^0) \cdot RT - U_j^0\cdot R}{(RT)^2} - \frac{\partial_T(S_j^0)}{R}\right)
    \label{dmu0_U}
\end{equation}

Recalling $\partial_T (S_j^0)$ from equation [\ref{dSdT}] and using:\

\begin{equation}
    c_{v,j}^0 = \frac{\partial U_j^0}{\partial T}
\end{equation}

Equation [\ref{dmu0_U}] reduced to:

\begin{equation}
    T \left[ \frac{c_{v,j}^0}{RT} - \frac{U_j^0}{RT^2} - \frac{c_{p,j}^0}{RT} \right]
\end{equation}

Since:

\begin{equation}
    \frac{c_{v,j}^0}{R} - \frac{c_{p.j}^0}{R} = -1
\end{equation}

The final derivative is:

\begin{equation}
    \boxed{ \frac{\partial}{\partial [\ln(T)]} \left( \frac{\mu_j^0}{RT} \right) = -\frac{U_j^0}{RT} - 1}
\end{equation}

\section{Coding Tricks}




\end{flushleft}
\end{document}