\documentclass[10pt]{article}

% Core math & graphics
\usepackage{amsmath}
\usepackage{graphicx}
\usepackage[dvipsnames]{xcolor} % load once, with options

% Floats, tables, layout
\usepackage{wrapfig}
\usepackage{tabularx}
\usepackage{array}
\usepackage{booktabs}

% Units/symbol helpers (see notes about gensymb below)
\usepackage{gensymb} % OK; if conflicts with \micro or \perthousand, consider siunitx

% Page geometry — choose margins OR total area (here: margins)
\usepackage[
  top=20mm,
  bottom=25mm,
  left=15mm,
  right=15mm
]{geometry}

% Captions/subcaptions (caption options applied before subcaption)
\usepackage[labelfont=bf]{caption}
\usepackage{subcaption}

% Misc utilities
\usepackage[export]{adjustbox}
\usepackage{setspace}
\usepackage{listings} % load once
\usepackage{mdframed}
\usepackage{tikz}
\usepackage{url}
\usepackage{marginnote}
\usepackage{csquotes}
\usepackage[normalem]{ulem} % keep \emph as italics
\usepackage{multicol}
\usepackage{lipsum}
\setlength{\parindent}{20pt}

% Number equations by section
\numberwithin{equation}{section}

% Nomenclature
\usepackage[intoc]{nomencl}
\makenomenclature

% Hyperlinks (keep near the end)
\usepackage{hyperref}
\hypersetup{
  colorlinks,
  citecolor=black,
  filecolor=black,
  linkcolor=black,
  urlcolor=black
}




\begin{document}


\begin{flushleft}

\onehalfspacing
\tableofcontents
\newpage

\nomenclature{$\mathcal{N}$}{Total number of moles in the system}
\nomenclature{$\mathcal{N}_j$}{Number of moles of species $j$ in the system}
\nomenclature{$\mathcal{L}$}{The Lagrangian}
\nomenclature{$a_{ij}$}{Stoichiometric coefficients, e.g. number of atoms of element $i$ in species $j$}
\nomenclature{$\hat{R}$}{Universal gas constant $\approx$ 8.314 J/kg-K}
\nomenclature{$b_i$}{Sum of contributions from each species towards constraint for element $i$, i.g. $\sum_{j=1}^\text{NS} a_{i,j}\mathcal{N}_j$}
\nomenclature{$p_\text{ref}$}{Reference temperature used to compute chemical potential}
\nomenclature{$G$}{Gibbs free-energy}
\nomenclature{$F$}{Helmholtz energy}
\nomenclature{$T$}{Temperature (K)}
\nomenclature{$q_j$}{Charge of species $j$ (0 for neutrals, 1 for ions, -1 for electrons)}
\nomenclature{$V$}{Volume ($m^3$)}
\nomenclature{$\rho$}{Density ($kg/m^3$}
\nomenclature{$MW_j$}{Molecular weight of species $j$}
\nomenclature{$\lambda$}{Lagrange multiplier}
\nomenclature{$\mu_j$}{Chemical potential of species $j$}
\nomenclature{$\mu_j^\circ$}{Standard state chemical potential of species $j$}
\nomenclature{$b_i^\circ$}{Specified number of moles of element $i$}
\nomenclature{$U_j^\circ$}{Standard state internal energy of species $j$}
\nomenclature{$H_j^\circ$}{Standard state enthalpy of species $j$}.
\nomenclature{$c_{v,j}^\circ$}{Standard state specific heat of species $j$}
\nomenclature{$c_{p,j}^\circ$}{Standard state specific heat of species $j$}
\nomenclature{$u_0'$}{User-specified internal energy}
\nomenclature{$h_0'$}{User-specified enthalpy}
\nomenclature{$u'$}{Sum of contributions from each species towards internal energy constraint, i.g. $\sum_{j}^\text{NS} \mathcal{N}_j U_j^\circ$}
\nomenclature{$h'$}{Sum of contributions from each species towards enthalpy  constraint, i.g. $\sum_{j}^\text{NS} \mathcal{N}_j H_j^\circ$}


\begin{multicols}{2}
\printnomenclature    
\end{multicols}


\section{Minimization Procedures}

\subsection{Constraints for Both Procedures}

There are two constraints that are used the most in both energy minimization procedures. The first, equation [\ref{elementalconstraint}], is always used while the charge constraints in equation [\ref{chargeconstraint}] is only used when ions are present:

\begin{equation}
   \sum_{j=1}^\text{NS} a_{ij} \mathcal{N}_j -  b_i^\circ= 0
    \label{elementalconstraint}
\end{equation}

\begin{equation}
    \sum_{j=1}^\text{NS}q_j \mathcal{N}_j = 0
    \label{chargeconstraint}
\end{equation}

Since these are used in abundance for both procedures, their constraints are placed in the Lagrangian term. Other constraints (such as specified volume, temperature, internal energy, and entropy) are given their rows in the Newton solve without addition into the Lagrangian. 

\vspace{5mm} 

Equation [$\ref{elementalconstraint}$] says that the number of moles of element $i$ must remain constant during the minimization process, as atoms can not be created or destroyed without nuclear physics being involved. $b_i^\circ$ is the number of moles of element $i$ before minimizing, and $a_{ij}$ is the stoichiometric coefficient, i.e. how many atoms of element $i$ are in species $j$. 

Equation [$\ref{chargeconstraint}$] says that the charge of the system must remain neutral, i.e., if cations are formed, than there must be an equal number of free electrons in the gas to balance it out.

\subsection{The Lagrangian}
The Lagrangian is defined as;

\begin{equation}
    \mathcal{L} = f + \sum_j \lambda_j g_j
    \label{Lagrangian}
\end{equation}

where $f$ is the function we want to minimize, $\lambda_j$ are Lagrange multipliers, and $g_j$ are constraint functions. Instead of minimizing just $f$, we aim to minimize the Lagrangian $\mathcal{L}$.

\subsection{Newton-Raphson Iterative Method}

A Newton-Raphson iterative procedure is used to minimize the energies. Since this is essentially a root finding system of equations, and we are seeking the solution update, the method looks like the following:

\begin{equation}
    \sum_{i=1}^N \frac{\partial f}{\partial x_i} \Delta x_i = -f(x)
    \label{newton}
\end{equation}

Here, $f(x)$ is the function that is being minimized. In the case of Gibbs and Helmholtz energy minimization, $f$ takes the form of $\partial \mathcal{L} / \partial x$, with $x$ being non-linear variables used to guide the convergence of the system. The non-linear correction variables are in the form $\Delta \ln \mathcal{N}_j$, $\Delta \ln \mathcal{N}$, $\Delta \ln T$, and $\pi_{i,q} = \Delta \pi_{i,q}$. For the last correction variable, it is argued that setting $\pi$ to 0 at the beginning of each update does not influence the solution for the non-reduced equations, so the $\Delta$ is dropped for that term. This will influence the form of the update equations, which will be talked about in due time.

\subsection{Gibbs Minimization}

The Gibbs energy is a function of temperature, pressure, and composition of a gas:

\begin{equation}
    G = G(T, p, \mathcal{N}_1, \mathcal{N}_2, ... , \mathcal{N}_\text{NS}) = \sum_{j=1}^\text{NS}\mathcal{N}_j \mu_j
    \label{gibbs}
\end{equation}

Where NS is the number of gas species, $\mathcal{N}_j$ is the number of moles of species $j$, and $\mu_j$ is the chemical potential of species $j$ defined by:

\begin{equation}
    \mu_j = \frac{\partial G}{\partial \mathcal{N}_j} = \mu_j^\circ + \hat{R}T \left[\ln\left(\frac{p}{p_\text{ref}}\right) + \ln \left( \frac{\mathcal{N}_j}{\mathcal{N}} \right) \right]
    \label{chempot}
\end{equation}

Here, $\hat{R}$ is the universal gas constant, $T$ and $p$ are the temperature and pressure of the system, $p_\text{ref}$ is the reference pressure (usually takes as 101,325 Pa, or 1 bar depending on the literature), and $\mathcal{N}$ is the total number of moles in the system. In order to find chemical equilibrium, the derivative of $G$ wrt $\mathcal{N}_j$ is set to 0 for all species. However, a number of constraints need to be added in. 

\subsubsection{Constraints}

Firstly, the elemental constraint from equation [\ref{elementalconstraint}] is added into the Lagrangian for our Gibbs function. If ions are present, the charge constraint is also put in. Therefore, our Lagrangian is:

\begin{equation}
    \mathcal{L} = \sum_{j=1}^\text{NS} \mu_j\mathcal{N}_j + \sum_{i=1}^\text{NE}\lambda_i 
    \left[\sum_{j=1}^\text{NS}a_{ij} \mathcal{N}_j - b_i^\circ\right] + \lambda_q \left[ \sum_{j = 0}^\text{NS}q_j\mathcal{N}_j \right]
\end{equation}

The contraints available for the Gibbs energy are:

\begin{itemize}
\item Hold temperature (T) and pressure (P) constant.
\item Hold enthalpy (H) and pressure (P) constant.
\item Hold entropy (S) and pressure (P) constant.
\end{itemize}

For all Gibbs minimizations, we need the molar constraint:

\begin{equation}
    \sum_{j=1}^\text{NS} \mathcal{N}_j - \mathcal{N} = 0
\end{equation}


For (TP) minimization, only the elemental and charge (if present) constraints are necessary. For (HP) minimization, the constraint is defined by:

\begin{equation}
    h' - h_0' = 0
\end{equation}

Where $h_0'$ is the specified enthalpy, and $h'$ is solved for during the minimization process as:

\begin{equation}
    h' = \sum_{j=1}^\text{NS} \mathcal{N}_j H_j^\circ
\end{equation}

For (SP) minimization, the constraint is defined by:

\begin{equation}
    s' - s_0' = 0
\end{equation}

Where $s_0'$ is the specified entropy and $s'$ is solved for in the minimization process as:

\begin{equation}
    s' = \sum_{j=1}^\text{NS} \mathcal{N}_j S_j
\end{equation}

where:

\begin{equation}
    S_j = S_j^\circ - \hat{R} \ln \frac{\mathcal{N}_j}{\mathcal{N}} - \hat{R} \ln p
\end{equation}

\subsubsection{Derivatives}

We now take the derivatives of $\mathcal{L}$ wrt $\mathcal{N}_j$, and each $\lambda$ and denote them as $f$:

\begin{align}
    f_1 &= \frac{\partial \mathcal{L}}{\partial \mathcal{N}_j} = \mu_j + \sum_{i=1}^\text{NE}\lambda_i a_{ij} + \lambda_q q_j = 0
    \label{f1}
    \\ f_2 &= \frac{\partial \mathcal{L}}{\partial \lambda_i} =  \sum_{i=1}^\text{NS} a_{ij} \mathcal{N}_j - b_i = 0
    \label{f2}
    \\ f_3 &= \frac{\partial \mathcal{L}}{\partial \lambda_q} = \sum_{j=0}^\text{NS}q_j\mathcal{N}_j = 0
    \label{f3}
\end{align}

These three equations are our minimization functions. Through use of the aforementioned Newton-Raphson method, we can solve these equations. Constraint equations for enthalpy and entropy are:

\begin{align}
    f_4 &= \sum_{j=1}^\text{NS} \mathcal{N}_j - \mathcal{N} = 0
    \\ f_5 &= \frac{h' - h_0'}{\hat{R}T} = \frac{\sum_{j=1}^\text{NS} \mathcal{N}_j H_j^\circ - h_0'}{\hat{R}T}
    \\ f_6 &= \frac{s' - s_0'}{\hat{R}} = \frac{\sum_{j=1}^\text{NS} \mathcal{N}_j S_j - s_0'}{\hat{R}}
\end{align}

\subsubsection{Derivatives of $f_1$}

First, we must do some algebraic manipulation. We start by expanding equation \ref{f1} with equation \ref{chempot}:

\begin{equation}
    \frac{\partial \mathcal{L}}{\partial \mathcal{N}_j} = \mu_j^\circ + \hat{R}T \left[\ln\left(\frac{p}{p_\text{ref}}\right) + \ln \left( \frac{\mathcal{N}_j}{\mathcal{N}} \right) \right] + \sum_{i=1}^\text{NE}\lambda_i a_{ij} + \lambda_q q_j = 0
\end{equation}

Dividing through by $\hat{R}T$ and setting $\pi = -\lambda/\hat{R}T$ gives:

\begin{equation}
    f_1 = \frac{\mu_j^\circ}{\hat{R}T} + \ln{\frac{p}{p_\text{ref}}} + ln{\mathcal{N}_j} - \ln{\mathcal{N}} - \sum_{i=1}^\text{NE}\pi_ia_{i,j} - \pi_q q_j = 0
    \label{f11}
\end{equation}


We now need to take the partial derivatives of our functions wrt our non-linear variables as well as each $\pi$, and then multiple it by the correction variables.

\begin{align}
    \frac{\partial f_1}{\partial[\ln{\mathcal{N}_j}]} \Delta\ln{\mathcal{N}_j} &= \Delta\ln{\mathcal{N}_j}
    \\ \frac{\partial f_1}{\partial[\ln{\mathcal{N}}]} \Delta\ln{\mathcal{N}} &= -\Delta\ln{\mathcal{N}}
    \\ \frac{\partial f_1}{\partial [\ln T]} \Delta \ln T &= - \frac{H_j^\circ}{\hat{R}T} \Delta \ln T
    \\ \frac{\partial f_1}{\partial [\pi_i]} \pi_i &= -a_{ij} \pi_i
    \\ \frac{\partial f_1}{\partial [\pi_q]} \pi_q &= -q_j \pi_q
\end{align}

Combining these into the form of equation \ref{newton} gives

\begin{equation}
    \Delta\ln{\mathcal{N}_j} - \Delta \ln{\mathcal{N}} - \sum_{i=1}^\text{NE} a_{ij}\pi_i - q_j \pi_q - \frac{H_j^\circ}{\hat{R}T} \Delta \ln T = \mu_j + \sum_{i=1}^\text{NE} \pi_i a_{ij} + \pi_q q_j
\end{equation}

Arguments are made that $\pi_i$ = 0 at the start of every iteration, so this equation becomes:


\begin{equation*}
    \boxed{\Delta\ln{\mathcal{N}_j} - \Delta \ln{\mathcal{N}} - \sum_{i=1}^\text{NE} a_{ij}\pi_i - q_j \pi_q -  \frac{H_j^\circ}{\hat{R}T} \Delta \ln T = -\frac{\mu_j}{\hat{R}T}}
\end{equation*}

There will be one of these equation for each species.  

\subsubsection{Derivatives of $f_2$}

\begin{align}
    \frac{\partial f_2}{\partial[\ln \mathcal{N}_j]} \Delta\ln\mathcal{N}_j &=   a_{ij} \mathcal{N}_j \Delta\ln{\mathcal{N}_j}
    \\ \frac{\partial f_2}{\partial [\ln \mathcal{N}]} \Delta \ln \mathcal{N} &= 0
    \\ \frac{\partial f_2}{\partial [\ln T]} &= 0
    \\ \frac{\partial f_2}{\partial [\pi_i]} \pi_i &= 0
    \\ \frac{\partial f_2}{\partial [\pi_q]} \pi_q &= 0
\end{align}

Which gives:

\begin{equation*}
    \boxed{\sum_{j=1}^\text{NS} a_{ij} \mathcal{N}_j \Delta \ln \mathcal{N}_j = b_i^\circ - \sum_{j=1}^\text{NS} a_{ij} \mathcal{N}_j}
\end{equation*}

\subsubsection{Derivatives of $f_3$}

\begin{align}
    \frac{\partial f_3}{\partial [\ln{\mathcal{N}_j}]} \Delta \ln \mathcal{N}_j &=  q_j \mathcal{N}_j \Delta\ln{\mathcal{N}_j}
    \\ \frac{\partial f_3}{\partial [\ln \mathcal{N}]} \Delta\ln \mathcal{N} &= 0
    \\ \frac{\partial f_3}{\partial [\ln T]} &= 0
    \\ \frac{\partial f_3}{\partial [\pi_i]} \pi_i &= 0
    \\ \frac{\partial f_3}{\partial [\pi_q]} \pi_q &= 0
\end{align}

Giving:

\begin{equation*}
    \boxed{\sum_{j=1}^\text{NS} q_j \mathcal{N}_j \Delta \ln \mathcal{N}_j = - \sum_{j=1}^\text{NS} q_i \mathcal{N}_j}
\end{equation*}

\subsubsection{Derivatives of $f_4$}

\begin{align}
    \frac{\partial f_4}{\partial [\ln{\mathcal{N}_j}]} \Delta \ln \mathcal{N}_j &= \mathcal{N}_j \Delta \ln \mathcal{N}_j 
    \\ \frac{\partial f_4}{\partial [\ln \mathcal{N}]} \Delta\ln \mathcal{N} &= -\mathcal{N} \Delta \ln \mathcal{N} 
    \\ \frac{\partial f_4}{\partial [\ln T]} &= 0
    \\ \frac{\partial f_4}{\partial [\pi_i]} \pi_i &= 0
    \\ \frac{\partial f_4}{\partial [\pi_q]} \pi_q &= 0
\end{align}

\begin{equation}
    \boxed{\sum_{j=1}^\text{NS} \mathcal{N}_j \Delta \ln \mathcal{N}_j - \mathcal{N} \Delta \ln \mathcal{N} = \mathcal{N} - \sum_{j=1}^\text{N} \mathcal{N}_j}
\end{equation}


\subsubsection{Derivatives of $f_5$}

\begin{align}
    \frac{\partial f_5}{\partial [\ln{\mathcal{N}_j}]} \Delta \ln \mathcal{N}_j &= \mathcal{N}_j \frac{H_j^\circ}{\hat{R}T} \Delta \ln \mathcal{N}_j  
    \\ \frac{\partial f_5}{\partial [\ln \mathcal{N}]} \Delta\ln \mathcal{N} &= 0
    \\ \frac{\partial f_5}{\partial [\ln T]} &=  \sum_{j=1}^\text{NS} \mathcal{N}_j \frac{c_{p,j}^\circ}{\hat{R}} \Delta \ln T
    \\ \frac{\partial f_5}{\partial [\pi_i]} \pi_i &= 0
    \\ \frac{\partial f_5}{\partial [\pi_q]} \pi_q &= 0
\end{align}

Giving:

\begin{equation*}
    \boxed{\sum_{j=1}^\text{NS} \mathcal{N}_j \frac{H_j^\circ}{\hat{R}T} \Delta \ln \mathcal{N}_j +  \sum_{j=1}^\text{NS} \mathcal{N}_j \frac{c_{p,j}^\circ}{\hat{R}} \Delta \ln T = \frac{h' - h_0'}{\hat{R}T}}
\end{equation*}

\subsubsection{Derivatives of $f_6$}

\begin{align}
    \frac{\partial f_6}{\partial [\ln{\mathcal{N}_j}]} \Delta \ln \mathcal{N}_j &= \mathcal{N}_j \frac{S_j}{\hat{R}} \Delta \ln \mathcal{N}_j
    \\ \frac{\partial f_6}{\partial [\ln \mathcal{N}]} \Delta\ln \mathcal{N} &= 0
    \\ \frac{\partial f_6}{\partial [\ln T]} &= \sum_{j=1}^\text{NS} \mathcal{N}_j \frac{c_{p,j}^\circ}{\hat{R}} \Delta \ln T
    \\ \frac{\partial f_6}{\partial [\pi_i]} \pi_i &= 0
    \\ \frac{\partial f_6}{\partial [\pi_q]} \pi_q &= 0
\end{align}

Giving

\begin{equation*}
    \boxed{\sum_{j=1}^\text{NS} \mathcal{N}_j \frac{S_j}{\hat{R}} \Delta \ln \mathcal{N}_j + \sum_{j=1}^\text{NS} \mathcal{N}_j \frac{c_{p,j}^\circ}{\hat{R}} \Delta \ln T = \frac{s_0' - s'}{\hat{R}} + \mathcal{N} - \sum_{j=1}^\text{NS} \mathcal{N}_j }
\end{equation*}

All five of the equations are now listed for convenience:

\begin{align}
    &\Delta\ln{\mathcal{N}_j} - \Delta \ln{\mathcal{N}} - \sum_{i=1}^\text{NE} a_{ij}\pi_i - q_j \pi_q -  \frac{H_j^\circ}{\hat{R}T} \Delta \ln T = -\frac{\mu_j}{\hat{R}T}
    \label{GlnNj}
    \\ &\sum_{j=1}^\text{NS} \mathcal{N}_j \Delta \ln \mathcal{N}_j - \mathcal{N} \Delta \ln \mathcal{N} = \mathcal{N} - \sum_{j=1}^\text{NS} \mathcal{N}_j
    \label{GlnN}
    \\ &\sum_{j=1}^\text{NS} a_{ij} \mathcal{N}_j \Delta \ln \mathcal{N}_j = b_i^\circ - \sum_{j=1}^\text{NS} a_{ij} \mathcal{N}_j
    \label{Gpii}
    \\ &\sum_{j=1}^\text{NS} q_j \mathcal{N}_j \Delta \ln \mathcal{N}_j = - \sum_{j=1}^\text{NS} q_i \mathcal{N}_j
    \label{Gpiq}
    \\ &\sum_{j=1}^\text{NS} \mathcal{N}_j \frac{H_j^\circ}{\hat{R}T} \Delta \ln \mathcal{N}_j +  \sum_{j=1}^\text{NS} \mathcal{N}_j \frac{c_{p,j}^\circ}{\hat{R}} \Delta \ln T = \frac{h_0' - h'}{\hat{R}T}
    \label{GdlnTH}
    \\ &\sum_{j=1}^\text{NS} \mathcal{N}_j \frac{S_j}{\hat{R}} \Delta \ln \mathcal{N}_j + \sum_{j=1}^\text{NS} \mathcal{N}_j \frac{c_{p,j}^\circ}{\hat{R}} \Delta \ln T = \frac{s_0' - s'}{\hat{R}} + \mathcal{N} - \sum_{j=1}^\text{NS} \mathcal{N}_j 
    \label{GdlnTS}
\end{align}


\subsubsection{Reduced Gibbs Equations}

A shortcut can be made to make this system of equations smaller. This is achieved by solving equation [\ref{GlnNj}] for $\Delta \ln \mathcal{N}_j$ and pluggin it into equations [\ref{Gpii}] - [\ref{GdlnTS}]

\begin{equation}
    \textcolor{red}{ \Delta\ln{\mathcal{N}_j} = \Delta \ln{\mathcal{N}} + \sum_{i=1}^\text{NE} a_{ij} \pi_i -\frac{\mu_j}{\hat{R}T} } \textcolor{blue}{+ q_j \pi_q} \textcolor{violet}{+ \frac{H_j^\circ}{\hat{R}T} \Delta \ln T}
    \label{GdLnNj}
\end{equation}


Substitution into equation [\ref{Gpii}] - [\ref{GdlnTS}] yields:

\begin{multline}
    \textcolor{red}{\sum_{i=1}^\text{NE} \left[ \sum_{j=1}^\text{NS} a_{kj} a_{ij} \mathcal{N}_j \right] \pi_i + \left[ \sum_{j=1}^\text{NS} a_{kj} \mathcal{N}_j \right] \Delta \ln \mathcal{N}} \textcolor{blue}{ + \left[ \sum_{j=1}^\text{NS} a_{kj} q_j \mathcal{N}_j \right] \pi_q} \textcolor{violet}{+ \left[ \sum_{j=1}^\text{NS} a_{kj} \mathcal{N}_j \frac{H_j^\circ}{\hat{R}T} \right] \Delta \ln T}
    \\ = \textcolor{red}{b_k^\circ - \sum_{j=1}^\text{NS} a_{kj} \mathcal{N}_j + \sum_{j=1}^\text{NS} a_{kj} \mathcal{N}_j \frac{\mu_j}{\hat{R}T}}
    \label{GRed}
\end{multline}

\begin{multline}
    \textcolor{red}{\sum_{i=1}^\text{NE} \left[ \sum_{j=1}^\text{NS} a_{ij} \mathcal{N}_j \right] \pi_i + \left[ \sum_{j=1}^\text{NS} \mathcal{N}_j - \mathcal{N} \right] \Delta \ln \mathcal{N}} \textcolor{blue}{ + \left[ \sum_{j=1}^\text{NS} q_j \mathcal{N}_j \right] \pi_q} \textcolor{violet}{+ \left[ \sum_{j=1}^\text{NS} \mathcal{N}_j \frac{H_j^\circ}{\hat{R}T} \right] \Delta \ln T}
    \\ = \textcolor{red}{\mathcal{N} - \sum_{j=1}^\text{NS} \mathcal{N}_j + \sum_{j=1}^\text{NS} \mathcal{N}_j \frac{\mu_j}{\hat{R}T}}
    \label{GRed}
\end{multline}


\begin{multline}
    \textcolor{blue}{\sum_{i=1}^\text{NE} \left[ \sum_{j=1}^\text{NS} a_{ij} q_j \mathcal{N}_j \right] \pi_i + \left[ \sum_{j=1}^\text{NS} q_j \mathcal{N}_j \right] \Delta \ln \mathcal{N} + \left[ \sum_{j=1}^\text{NS} q_j^2 \mathcal{N}_j \right] \pi_q} + \textcolor{violet}{\left[ \sum_{j=1}^\text{NS} q_j \mathcal{N}_j \frac{H_j^\circ}{\hat{R}T} \right] \Delta \ln T}
    \\ = \textcolor{blue}{\sum_{j=1}^\text{NS} q_j \mathcal{N}_j \frac{\mu_j}{\hat{R}T} - \sum_{j=1}^\text{NS} q_j \mathcal{N}_j}
\end{multline}


\begin{multline}
    \textcolor{orange}{\sum_{i=1}^\text{NE} \left[ \sum_{j=1}^\text{NS} a_{ij} \mathcal{N}_j \frac{H_j^\circ}{\hat{R}T} \right] \pi_i + \left[ \sum_{j=1}^\text{NS} \mathcal{N}_j \frac{H_j^\circ}{\hat{R}T} \right] \Delta \ln \mathcal{N}} \textcolor{blue}{+ \left[ \sum_{j=1}^\text{NS} q_j \mathcal{N}_j \frac{H_j^\circ}{\hat{R}T} \right] \pi_q} 
    \\ \textcolor{orange}{+ \left[ \sum_{j=1}^\text{NS} \mathcal{N}_j \left( \frac{H_j^\circ}{\hat{R}T} \right)^2 + \sum_{j=1}^\text{NS} \mathcal{N}_j \frac{c_{p,j}^\circ}{\hat{R}} \right] \Delta \ln T = \frac{h_0' - h'}{\hat{R}T} + \sum_{j=1}^\text{NS} \mathcal{N}_j \frac{H_j^\circ}{\hat{R}T} \frac{\mu_j}{\hat{R}T}}
\end{multline}


\begin{multline}
    \textcolor{teal}{\sum_{i=1}^\text{NE} \left[ \sum_{j=1}^\text{NS} a_{ij} \mathcal{N}_j \frac{S_j}{\hat{R}} \right] \pi_i + \left[ \sum_{j=1}^\text{NS} \mathcal{N}_j \frac{S_j}{\hat{R}} \right] \Delta \ln \mathcal{N}} \textcolor{blue}{+ \left[ \sum_{j=1}^\text{NS} q_j \mathcal{N}_j \frac{S_j}{\hat{R}} \right] \pi_q} 
    \\ \textcolor{teal}{+ \left[ \sum_{j=1}^\text{NS} \mathcal{N}_j \frac{S_j}{\hat{R}} \frac{H_j^\circ}{\hat{R}T} + \sum_{j=1}^\text{NS} \mathcal{N}_j \frac{c_{p,j}^\circ}{\hat{R}} \right] \Delta \ln T = \frac{s_0' - s'}{\hat{R}} + \mathcal{N} - \sum_{j=1}^\text{NS} \mathcal{N}_j + \sum_{j=1}^\text{NS} \mathcal{N}_j \frac{S_j}{\hat{R}} \frac{\mu_j}{\hat{R}T} }
\end{multline}

There are NE number of equation [\ref{GRed}], and only one of all others. Once this system has been solved, you recover $\mathcal{N}_j$ with equation [\ref{GdLnNj}]. 
\vspace{5mm}
\\ The equations are colored according to when they are needed. \textcolor{red}{Red coloring} denotes terms that arise from the elemental constraint condition. These are included no matter what. \textcolor{blue}{Blue} denotes terms that are added when you include the charge constraint. \textcolor{violet}{Violet} terms are added for both HP and SP constraints. \textcolor{orange}{orange} and \textcolor{teal}{teal} are the rows added for enthalpy or entropy constraints, respectively.


\subsection{Helmholtz Minimization}

The Helmholtz energy is defined as:

\begin{equation}
    F = G - pV
\end{equation}

Where $G$ is the Gibbs free energy, $p$ is the pressure, and $V$ is the volume. Substitution of $G$ yields:

\begin{equation}
    F = \sum_{j=1}^\text{NS} \mu_j \mathcal{N}_j - pV
\end{equation}

Where the chemical potential is redefined as:

\begin{equation}
    \mu_j = \mu_j^\circ + \hat{R}T \ln\left( \frac{\mathcal{N}_j R'T}{V} \right)
    \label{chempotF}
\end{equation}

and $R' = \hat{R} \cdot 10^{-5}$ is the same as dividing $\hat{R}$ by the reference pressure of 1 bar. 

\subsubsection{Constraints}
The contraints available for the Helmholtz energy are:

\begin{itemize}
\item Hold temperature (T) and volume (V) constant.
\item Hold internal energy (U) and volume (V) constant.
\item Hold entropy (S) and volume (V) constant.
\end{itemize}

For (TV) minimization, only the elemental and charge (if present) constraints are necessary. For (UV) minimization, the constraint is defined by:

\begin{equation}
    u' - u_0' = 0
\end{equation}

Where $u_0'$ is the specified internal energy, and $u'$ is solved for during the minimization process as:

\begin{equation}
    u' = \sum_{j=1}^\text{NS} \mathcal{N}_j U_j^\circ = u_0'
\end{equation}

For (SV) minimization, the constraint is defined by:

\begin{equation}
    s' - s_0' = 0
\end{equation}

Where $s_0'$ is the specified entropy and $s'$ is solved for duging the minimization process as:

\begin{equation}
    s' = \sum_{j=1}^\text{NS} \mathcal{N}_j S_j
\end{equation}

where:

\begin{equation}
    S_j = S_j^\circ - \hat{R} \ln \mathcal{N}_j - \hat{R} \ln \left( \frac{R'T}{V}\right)
\end{equation}


While you may be able to include the constraints for internal energy and entropy inside the Lagrangian, we will exclude them from the Lagrangian and use our own residual equation for them. The Lagrangian for the Helmholtz energy minimization process is then defined as:

\begin{equation}
    \mathcal{L} = \sum_{j=1}^\text{NS} \mu_j \mathcal{N}_j - pV + \sum_{i=1}^\text{NE}\lambda_i \left[ \sum_{j=1}^\text{NS}a_{ij} \mathcal{N}_j - b_i^\circ \right] + \lambda_q \left[\sum_{j=1}^\text{NS} q_j \mathcal{N}_j\right]
    \label{helmholtzL}
\end{equation}

\subsubsection{Derivatives}
We first find our function $f$ that are equal to 0 by taking the derivatives wrt $\mathcal{N}_j,$ $\lambda_i$, and $\lambda_q$:

\begin{align}
    f_1 &= \frac{\partial \mathcal{L}}{\partial \mathcal{N}_j} = \mu_j + \sum_{i=1}^\text{NE}\lambda_ia_{ij} + \lambda_q q_j = 0
    \\ f_2 &= \frac{\partial \mathcal{L}}{\partial \lambda_i} = \sum_{j=1}^\text{NS} a_{ij} \mathcal{N}_j - b_i^\circ= 0
    \\ f_3 &= \frac{\partial \mathcal{L}}{\partial \lambda_q} = \sum_{j=1}^\text{NS} q_j \mathcal{N}_j - 0= 0
\end{align}

The negative of these functions ($-f$) are played on the RHS of the Newton solver.We now form the second derivatives that define our Jacobian in the Newton step. We expand the chemical potential term using equation \ref{chempotF} and non-dimensionalize the equation by $\hat{R}T$. We make sure to convert these equations into a form that can be differentiated by our non-linear variables (i.e. $N_j = \exp(\ln(\mathcal{N}_j))$). We also set $\pi = -\lambda/\hat{R}T$. 

\subsubsection{Derivatives for $f_1$}
The derivatives for $f_1$ are then:

\begin{align}
    \frac{\partial f_1}{\partial [\ln \mathcal{N}_j]} \Delta \ln \mathcal{N}_j &= \Delta \ln \mathcal{N}_j
    \\ \frac{\partial f_1}{\partial [\ln T]} \Delta \ln T &= -\frac{U_j^\circ}{\hat{R}T} \ln T
    \label{f1lnT}
    \\ \frac{\partial f_1}{\partial \pi_i} \pi_i &= -a_{ij} \pi_i
    \\ \frac{\partial f_1}{\partial \pi_q} \pi_q &= -q_j \pi_q
\end{align} 

The derivation for equation [\ref{f1lnT}] is carried out in more depth in section [\ref{derivativesF}]. Therefore, our final result is:

\begin{equation*}
    \boxed{\Delta \ln \mathcal{N}_j - \sum_{i=1}^\text{NE}a_{ij} \pi_i - q_j \pi_q -\frac{U_j^\circ}{\hat{R}T} \Delta \ln T  = -\frac{\mu_j}{\hat{R}T} }
\end{equation*}

There will be NS number of these equations in our system. 


\subsubsection{Derivatives for $f_2$}
Next, we take the derivatives of $f_2$:

\begin{align}
    \frac{\partial f_2}{\partial [\ln \mathcal{N}_j]} \Delta \ln \mathcal{N}_j &= \mathcal{N}_j a_{i,j} \Delta \ln \mathcal{N}_j
    \\ \frac{\partial f_2}{\partial [\ln T]} \Delta \ln T &= 0
    \\ \frac{\partial f_2}{\partial \pi_i} \pi_i &= 0
    \\ \frac{\partial f_2}{\partial \pi_q} \pi_q &= 0
\end{align} 

Which gives:

\begin{equation*}
    \boxed{ \sum_{j=1}^\text{NS}\mathcal{N}_j a_{ij} \Delta \ln \mathcal{N}_j = b_i^\circ - \sum_{j=1}^\text{NS} a_{ij} \mathcal{N}_j}
\end{equation*}

There will be NE of these equations. 

\subsubsection{Derivatives for $f_3$} 

For $f_3$ we get:

\begin{align}
    \frac{\partial f_3}{\partial [\ln \mathcal{N}_j]} \Delta \ln \mathcal{N}_j &= \mathcal{N}_j q_j \Delta \ln \mathcal{N}_j
    \\ \frac{\partial f_3}{\partial [\ln T]} \Delta \ln T &= 0
    \\ \frac{\partial f_3}{\partial \pi_i} \pi_i &= 0
    \\ \frac{\partial f_3}{\partial \pi_q} \pi_q &= 0
\end{align} 

Which gives only one equation:

\begin{equation*}
    \boxed{ \sum_{j=1}^\text{NS}\mathcal{N}_j q_j \Delta \ln \mathcal{N}_j = -\sum_{j=1}^\text{NS} q_j \mathcal{N}_j}
\end{equation*}


\subsubsection{Derivatives for $f_4$}
For internal energy, we have:

\begin{equation}
    f_4(\ln \mathcal{N}_j, \ln T) = \sum_{j=1}^\text{NS} \mathcal{N}_j U_j^\circ - u_0' = u' - u_0'
\end{equation}

Using identity [\ref{dTID}] helps with the temperature derivative. The derivatives are then:

\begin{align}
    \frac{\partial f_4}{\partial [\ln \mathcal{N}_j]} \Delta \ln \mathcal{N}_j &= \frac{\mathcal{N}_jU_j^\circ}{\hat{R}T} \Delta \ln \mathcal{N}_j
    \\ \frac{\partial f_4}{\partial [\ln T]} \Delta \ln T &= \sum_{j=1}^\text{NS} [T \mathcal{N}_jc_{v,j}^\circ] \Delta \ln T
    \\ \frac{\partial f_4}{\partial \pi_i} \pi_i &= 0
    \\ \frac{\partial f_4}{\partial \pi_q} \pi_q &= 0
\end{align}

After non-dimensionalizing, we are left with:

\begin{equation*}
    \boxed{ \sum_{j=1}^\text{NS}\frac{\mathcal{N}_jU_j^\circ}{\hat{R}T} \Delta \ln \mathcal{N}_j + \sum_{j=1}^\text{NS} \frac{ \mathcal{N}_jc_{v,j}^\circ}{R} \Delta \ln T = \frac{u_0' - u'}{\hat{RT}}}
\end{equation*}

\subsubsection{Derivatives for $f_5$}
For the entropy constraint we have:

\begin{equation}
    f_5(\ln \mathcal{N}_j, \ln T) = \sum_{j=1}^\text{NS} \mathcal{N}_j S_j - s_0' = s' - s_0'
\end{equation}

Again, identity [\ref{dTID}] assists us. The derivates are:

\begin{align}
    \frac{\partial f_5}{\partial [\ln \mathcal{N}_j]} \Delta \ln \mathcal{N}_j &= \mathcal{N}_j [S_j - \hat{R}] \Delta \ln \mathcal{N}_j
    \\ \frac{\partial f_5}{\partial [\ln T]} \Delta \ln T &= \mathcal{N}_j c_{v,j}^\circ \Delta \ln T
    \\ \frac{\partial f_5}{\partial \pi_i} \pi_i &= 0
    \\ \frac{\partial f_5}{\partial \pi_q} \pi_q &= 0
\end{align}

Which, after non-dimensionalizing gives the singular equation:

\begin{equation}
    \boxed{\sum_{j=1}^\text{NS} \mathcal{N}_j \left[ \frac{S_j}{\hat{R}} - 1 \right] \Delta \ln \mathcal{N}_j + \mathcal{N}_j \frac{c_{v,j}^\circ}{\hat{R}} \Delta \ln T = \frac{s_0' - s'}{\hat{R}}}
\end{equation}


For convenience, they are all listed here together:

\begin{align}
        &\Delta \ln \mathcal{N}_j - \sum_{i=1}^\text{NE}a_{ij} \pi_i - q_j \pi_q -\frac{U_j^\circ}{\hat{R}T} \Delta \ln T  = -\frac{\mu_j}{\hat{R}T} 
        \label{HdlnNj}
        \\ &\sum_{j=1}^\text{NS}\mathcal{N}_j a_{ij} \Delta \ln \mathcal{N}_j = b_i^\circ - \sum_{j=1}^\text{NS} a_{ij} \mathcal{N}_j
        \label{Hdpi}
        \\ &\sum_{j=1}^\text{NS}\mathcal{N}_j q_j \Delta \ln \mathcal{N}_j = -\sum_{j=1}^\text{NS} q_j \mathcal{N}_j
        \label{Hdpiq}
        \\ &\sum_{j=1}^\text{NS}\frac{\mathcal{N}_jU_j^\circ}{\hat{R}T} \Delta \ln \mathcal{N}_j + \sum_{j=1}^\text{NS} \frac{ \mathcal{N}_jc_{v,j}^\circ}{R} \Delta \ln T = \frac{u_0' - u'}{\hat{RT}}
        \label{HdlnT}
        \\ &\sum_{j=1}^\text{NS} \mathcal{N}_j \left[ \frac{S_j}{\hat{R}} - 1 \right] \Delta \ln \mathcal{N}_j + \mathcal{N}_j \frac{c_{v,j}^\circ}{\hat{R}} \Delta \ln T = \frac{s_0' - s'}{\hat{R}}
\end{align}

\subsubsection{Reduced Helmholtz Equations}

A "simplification" can be made to the above system of equations by solving equations [\ref{HdlnNj}] for $\Delta \ln \mathcal{N}_j$:

\begin{equation}
    \textcolor{red}{\Delta \ln \mathcal{N}_j =  \sum_{i=1}^\text{NE}a_{i,j}\pi_i -\frac{\mu_j}{\hat{R}T} } \textcolor{blue}{+ q_j \pi_q}\textcolor{violet}{ + \frac{U_j^\circ}{\hat{R}T} \Delta \ln T }
    \label{redH0}
\end{equation}

By substituting this into equations [\ref{Hdpi}] - [\ref{HdlnT}], we can shrink the system of equations dramatically. The resulting set of equations is:

\begin{equation}
    \textcolor{red}{\sum_{i=1}^\text{NE} \left[ \sum_{j=1}^\text{NS} a_{kj} a_{ij} \mathcal{N}_j \right] \pi_i} \textcolor{blue}{+ \left[\sum_{j=1}^\text{NS} a_{kj} q_j\mathcal{N}_j \right] \pi_q} \textcolor{violet}{+ \left[ \sum_{j=1}^\text{NS} a_{kj} \mathcal{N}_j\frac{ U_j^\circ}{\hat{R}T} \right] \Delta \ln T} = \textcolor{red}{b_k^\circ - \sum_{j=1}^\text{NS} a_{kj} \mathcal{N}_j +  \sum_{j=1}^\text{NS} a_{kj} \mathcal{N}_j\frac{ \mu_j}{\hat{R}T}}
    \label{redH1}
\end{equation}

\begin{equation}
    \textcolor{blue}{\sum_{i=1}^\text{NE} \left[ \sum_{j=1}^\text{NS} a_{ij} q_j \mathcal{N}_j \right] \pi_i + \left[ \sum_{j=1}^\text{NS}  q_j^2 \mathcal{N}_j \right] \pi_q} \textcolor{violet}{+ \left[ \sum_{j=1}^\text{NS} q_j \mathcal{N}_j\frac{ U_j^\circ}{\hat{R}T} \right] \Delta \ln T} = \textcolor{blue}{\sum_{j=1}^\text{NS} q_j \mathcal{N}_j \frac{\mu_j}{\hat{R}T} - \sum_{j=1}^\text{NS} q_j \mathcal{N}_j}
    \label{redH2}
\end{equation}

\begin{multline}
    \textcolor{orange}{\sum_{i=1}^\text{NE} \left[ \sum_{j=1}^\text{NS} a_{ij} \mathcal{N}_j\frac{ U_j^\circ}{\hat{R}T} \right] \pi_i} \textcolor{blue}{+ \left[ \sum_{j=1}^\text{NS} q_j \mathcal{N}_j \frac{ U_j^\circ}{\hat{R}T} \right] \pi_q} \textcolor{orange}{+ \left[ \sum_{j=1}^\text{NS} \mathcal{N}_j \left( \frac{ U_j^{\circ}}{\hat{R} T} \right)^2 + \sum_{j=1}^\text{NS} \mathcal{N}_j \frac{ c_{v,j}^\circ}{\hat{R}} \right] \Delta \ln T}
    \\ = \textcolor{orange}{\frac{u_0' - u'}{\hat{R}T} + \sum_{j=1}^\text{NS} \mathcal{N}_j \frac{ U_j^\circ}{\hat{R} T} \frac{\mu_j}{\hat{R}T}}
    \label{redH3}
\end{multline}

\begin{multline}
    \textcolor{teal}{\sum_{i=1}^\text{NE} \left[ \sum_{j=1}^\text{NS} a_{ij} \mathcal{N}_j \left( \frac{S_j}{\hat{R}} - 1 \right) \right] \pi_i} \textcolor{blue}{ + \left[\sum_{j=1}^\text{NS} q_j \mathcal{N}_j \left( \frac{S_j}{\hat{R}} - 1\right) \right] \pi_q} 
    \\ + \textcolor{teal}{\left[\sum_{j=1}^\text{NS} \mathcal{N}_j \left( \frac{S_j}{\hat{R}} - 1 \right) \frac{U_j^\circ}{\hat{R}T} + \sum_{j=1}^\text{NS} \mathcal{N}_j \frac{c_{v,j}^\circ}{\hat{R}} \right] \Delta \ln T  = \frac{s_0' - s'}{\hat{R}} + \sum_{j=1}^\text{NS} \mathcal{N}_j \frac{\mu_j}{\hat{R}T} \left( \frac{S_j}{\hat{R}} - 1 \right)} 
    \label{redH4}
\end{multline}


There will be NE number of equations $\ref{redH1}$, and one of $\ref{redH2}$ if charge is added, and one of $\ref{redH3}/\ref{redH4}$ depending on which one you made need.
\vspace{5mm}
\\ The equations are colored according to when they are needed. \textcolor{red}{Red coloring} denotes terms that arise from the elemental constraint condition. These are included no matter what. \textcolor{blue}{Blue} denotes terms that are added when you include the charge constraint. \textcolor{violet}{Violet} terms are added for both UV and SV constraints. \textcolor{orange}{orange} and \textcolor{teal}{teal} are the rows added for either internal energy or entropy constraints.

\subsection{Useful Relationships}

\begin{equation}
    \frac{c_v^\circ}{\hat{R}} = \frac{c_{p,j}^\circ}{\hat{R}} -1
\end{equation}

\begin{equation}
    \frac{\mu_j^\circ}{\hat{R}T} = \frac{H_j^\circ}{\hat{R}T} - \frac{S_j^\circ}{\hat{R}} 
\end{equation}

\begin{equation}
    \frac{U_j^\circ}{\hat{R}T} = \frac{H_j^\circ}{\hat{R}T} - 1
\end{equation}


\section{Univeristy of Minnesota Chemical Equilibrium Code UMCEC}

The University of Minnesota code for computing chemical equilibrium of a gas mixture uses the well documented method of Gibbs and Helmholtz energy minimization with Lagrange multipliers as well as charge constrains for plasmas, and enthalpy, entropy, and internal energy constraints depending on the minimization method. It computes the composition and thermodynamic state of gas mixtures. 



\section{Standard State Chemical Potential Derivatives}

Some important derivatives are derived here. First, we state a chain rule identity that is very useful:

\begin{equation}
    \frac{\partial}{\partial T} = \frac{\partial [\ln (T)]}{\partial T} \frac{\partial}{\partial [\ln(T)]} \rightarrow \frac{\partial}{\partial [\ln(T)]} = T \frac{\partial}{\partial T}
    \label{dTID}
\end{equation}

Then, taking the derivative of $\mu_j^0$ gives:

\begin{equation}
    \frac{\partial}{\partial [\ln(T)]} \left( \frac{\mu_j^0}{RT} \right)  = T \frac{\partial }{\partial T} \left( \frac{\mu_j^0}{RT} \right)
\end{equation}

We expand $\mu_j^0 /RT$ as:

\begin{align}
    \frac{\mu_j^0}{RT} &= \frac{H_j^0}{RT} - \frac{S_j^0}{R} 
    \label{mu0_H}
    \\ \frac{\mu_j^0}{RT} &= \frac{U_j^0}{RT} - \frac{S_j^0}{R} + 1
    \label{mu0_U}
\end{align}

\subsection{Gibbs Derivative}
\label{derivativesG}
First, we take the derivative of [\ref{mu0_H}] for Gibbs minimization.

\begin{equation}
    T \frac{\partial }{\partial T} \left( \frac{H_j^0}{RT} - \frac{S_j^0}{R} \right) = T \left( \frac{\partial_T(H_j^0) \cdot RT - H_j^0\cdot R}{(RT)^2} - \frac{\partial_T(S_j^0)}{R}\right)
     \label{dmu0_H}
\end{equation}

Here, $\partial_T$ denotes ($\partial/ \partial T$). If:

\begin{equation}
    c_{p,j}^0 = \frac{\partial H_j^0}{\partial T}, \hspace{2mm} \text{and} \hspace{2mm} \frac{c_{p,j}^0}{T} = \frac{\partial S_j^0}{\partial T} 
    \label{dSdT}
\end{equation}

This reduces to:

\begin{equation}
    T \left [  \frac{c_{p,j}^0}{RT} - \frac{H_j^0}{RT^2}  - \frac{c_{p,j}^0}{RT} \right] = -\frac{H_j^0}{RT}
\end{equation}

Therefore:

\begin{equation}
    \boxed{\frac{\partial}{\partial[\ln(T)]} \left( \frac{\mu_j^0}{RT} \right) = -\frac{H_j^0}{RT} }
\end{equation}

\subsection{Helmholtz Derivative}
\label{derivativesF}

Using equation [\ref{mu0_U}] gives us:

\begin{equation}
    T \frac{\partial }{\partial T} \left( \frac{U_j^0}{RT}  + 1- \frac{S_j^0}{R} \right) = T \left( \frac{\partial_T(U_j^0) \cdot RT - U_j^0\cdot R}{(RT)^2} - \frac{\partial_T(S_j^0)}{R}\right)
    \label{dmu0_U}
\end{equation}

Recalling $\partial_T (S_j^0)$ from equation [\ref{dSdT}] and using:\

\begin{equation}
    c_{v,j}^0 = \frac{\partial U_j^0}{\partial T}
\end{equation}

Equation [\ref{dmu0_U}] reduced to:

\begin{equation}
    T \left[ \frac{c_{v,j}^0}{RT} - \frac{U_j^0}{RT^2} - \frac{c_{p,j}^0}{RT} \right]
\end{equation}

Since:

\begin{equation}
    \frac{c_{v,j}^0}{R} - \frac{c_{p.j}^0}{R} = -1
\end{equation}

The final derivative is:

\begin{equation}
    \boxed{ \frac{\partial}{\partial [\ln(T)]} \left( \frac{\mu_j^0}{RT} \right) = -\frac{U_j^0}{RT} - 1}
\end{equation}

\newpage
\section{Matrix Method for computation of Jacobian matrix for CFD}

The reduced equations for Helmholtz energy minimization contains quite a large number of sums. These sums are necessary for every iteration since the sums contain $\mathcal{N}_j$, or some thermodynamic value like $H_j^\circ$. However, this can be reduced even further with the introduction of the $\boldsymbol{\xi}$ matrix, and a seperate matrix $\boldsymbol{\omega}$ which does not reduce computation time.
\vspace{5mm}
\\ Since each original Jacobian entry is essentially just the dot product of some vector with the $\mathcal{N}_j$ vector, we can create these matrices such that, when multiplied by the column vector $\hat{\mathcal{N}}_j$, give a resultant column vector that contains terms for the Jacobian matrix and RHS vector of the Newton-Raphson iteration. This does not speed up the computation by a lot, but it certainly gets rid of some redunancy as well as provididing some cache locality. The $\boldsymbol{\xi}$ matrix is defined as 

\begin{equation}
    \boldsymbol{\xi}(a_{ij}, q_j) \begin{bmatrix} \mathcal{N}_1 & \mathcal{N}_2 & \dots & \mathcal{N}_\text{NS} \end{bmatrix}^T = \boldsymbol{\mathcal{J}}(a_{ij}, q_j, \mathcal{N}_j)
\end{equation}

Where the variables in parenthese mean that that matrix or vector only contain sums with those variables. The $\boldsymbol{\omega}$ matrix is defined as:

\begin{equation}
    \boldsymbol{\omega}(a_{ij}, q_j, H_j^\circ) \begin{bmatrix} \mathcal{N}_1 & \mathcal{N}_2 & \dots & \mathcal{N}_\text{NS} \end{bmatrix}^T = \boldsymbol{\mathcal{J}}(a_{ij}, q_j, H_j^\circ, \mathcal{N}_j)
\end{equation}


These matrices will be shown for each section of the Jacobian. The number of columns of the $\boldsymbol{\xi}$ and $\boldsymbol{\omega}$ matrices is equal to the number of species of gas in the system.

\subsection{The $\xi$ matrix}

\subsubsection{Elemental Rows}

The following matrix creates the $k^\text{th}$ row of the Jacobian matrix, where $k$ varies from 0 to NE - 1. The left hand matrix denotes which entry of the Jacobian matrix that row is associated with. There will be NE number of these stacked on top of each other:

\begin{equation}
    \begin{vmatrix}
        J[k, 0]
        \\ \vdots
        \\ J[k, \text{NE - 1}]
        \\ J[k, \text{NE}]
    \end{vmatrix}
    =
    \begin{bmatrix}
    a_{k1}a_{11} & \dots & a_{k, \text{NS}} a_{1, \text{NS}}
    \\ \vdots & \ddots & \vdots &
    \\ a_{k,1} a_{\text{NE}, 1} & \dots & a_{k, \text{NS}} a_{\text{NE}, \text
    {NS}}
    \\ a_{k,1} q_1 & \dots & a_{k, \text{NS}} q_\text{NS}
    \end{bmatrix}
    \begin{bmatrix}
        \mathcal{N}_1
        \\ \vdots
        \\ \mathcal{N}_\text{NS}
    \end{bmatrix}
\end{equation}

\subsubsection{Charge rows}

For the charge constraint, we have:

\begin{equation}
    \begin{vmatrix}
        J[\text{NE}, 0]
        \\ \vdots
        \\ J[\text{NE}, \text{NE - 1}]
        \\ J[\text{NE}, \text{NE}]
    \end{vmatrix}
    =
    \begin{bmatrix}
    a_{11} q_1 & \dots & a_{1, \text{NS}} q_\text{NS} 
    \\ \vdots & \ddots & \vdots &
    \\ a_{\text{NE}, 1} q_1 & \dots & a_{\text{NE}, \text{NS}} q_\text{NS}
    \\ q_1^2 & \dots & q_\text{NS}^2
    \end{bmatrix}
    \begin{bmatrix}
        \mathcal{N}_1
        \\ \vdots
        \\ \mathcal{N}_\text{NS}
    \end{bmatrix}
\end{equation}

The elemental rows has already shown up in the previous section, so we can reuse its values and therefore the only additional row is only:

\begin{equation}
    \begin{vmatrix}
        J[\text{NE}, \text{NE}]
    \end{vmatrix}
    =
    \begin{bmatrix}
    q_1^2 & \dots & q_\text{NS}^2
    \end{bmatrix}
    \begin{bmatrix}
        \mathcal{N}_1
        \\ \vdots
        \\ \mathcal{N}_\text{NS}
    \end{bmatrix}
\end{equation}

With:

\begin{equation}
    \begin{vmatrix}
        J[\text{NE}, 0]
        \\ \vdots
        \\ J[\text{NE}, \text{NE} - 1]
    \end{vmatrix}
    =
    \begin{vmatrix}
        J[0, \text{NE}]
        \\ \vdots
        \\ J[\text{NE} - 1, \text{NE}]
    \end{vmatrix}
\end{equation}


\subsubsection{RHS}

At the end of our $\xi$ matrix, we place two terms that arrive on the RHS, namely the dot produces of $\mathcal{N}_j$ with elemental and charges. Therefore we get:

\begin{equation}
    \begin{vmatrix}
        \text{RHS}[\text{i}]
        \\ \vdots
        \\ \text{RHS}[\text{NE - 1}]
        \\ \text{RHS}[\text{NE}]
    \end{vmatrix}
    =
    \begin{bmatrix}
    a_{11} & \dots & a_{1, \text{NS}}
    \\ \vdots & \ddots & \vdots &
    \\ a_{\text{NE}, 1} & \dots & a_{\text{NE}, \text{NS}}
    \\ q_1 & \dots & q_\text{NS}
    \end{bmatrix}
    \begin{bmatrix}
        \mathcal{N}_1
        \\ \vdots
        \\ \mathcal{N}_\text{NS}
    \end{bmatrix}
\end{equation}


\subsubsection{Full $\xi$ matrix}

The full size of the $\xi$ matrix is $[\text{NE}^2 + \text{NE} + \eta(\text{NE} + 2)] \cdot \text{NS}$, where $\eta$ equals 0 for no ionization, and 1 for ionization. For 5-species air, this is a 30 entry matrix. For 11 species air with 2 elements, this is a 111 element matrix. Here, all charge entries have been brough to the bottom for easy indexing.

\begin{equation}
    \boldsymbol{\xi} \mathcal{N} = 
    \begin{bmatrix}
        a_{k1}a_{11} & \dots & a_{k, \text{NS}} a_{1, \text{NS}}
        \\ \vdots & \ddots & \vdots
        \\ a_{k,1} a_{\text{NE}, 1} & \dots & a_{k, \text{NS}} a_{\text{NE}, \text{NS}}
        \\ \vdots & \ddots & \vdots
        \\ a_{\text{NE}, 1}a_{11} & \dots & a_{\text{NE}, \text{NS}} a_{1, \text{NS}}
        \\ \vdots & \ddots & \vdots &
        \\ a_{\text{NE}, 1} a_{\text{NE}, 1} & \dots & a_{\text{NE}, \text{NS}} a_{\text{NE}, \text{NS}}
        \\ a_{11} & \dots & a_{1, \text{NS}}
        \\ \vdots & \ddots & \vdots &
        \\ a_{\text{NE}, 1} & \dots & a_{\text{NE}, \text{NS}}
        \\ a_{k,1} q_1 & \dots & a_{k, \text{NS}} q_\text{NS}
        \\ \vdots & \ddots & \vdots &
        \\ a_{\text{NE}, 1} q_1 & \dots & a_{\text{NE}, \text{NS}} q_\text{NS}
        \\ q_1 & \dots & q_\text{NS}
        \\ q_1^2 & \dots & q_\text{NS}^2
    \end{bmatrix}    
    \begin{bmatrix}
        \mathcal{N}_1
        \\ \vdots
        \\ \mathcal{N}_i
        \\ \vdots
        \\ \mathcal{N}_\text{NS}
    \end{bmatrix}
    =
    \begin{bmatrix}
        \text{RHS}[k \cdot \text{NE}]
        \\ \vdots
        \\ \text{RHS}[k \cdot \text{NE} + (NE - 1)]
        \\ \vdots
        \\ \text{RHS}[(\text{NE} - 1) \text{NE}]
        \\ \vdots
        \\ \text{RHS}[(\text{NE} - 1) \text{NE} + \text{NE} - 1]
        \\ \text{RHS}[\text{NE} \cdot \text{NE}]
        \\ \vdots
        \\ \text{RHS}[\text{NE} \cdot \text{NE} + \text{NE} - 1]
        \\ \text{RHS}[\text{NE} \cdot (\text{NE} + 1)]
        \\ \vdots
        \\ \text{RHS}[\text{NE} \cdot (\text{NE} + 1) + \text{NE} - 1]
        \\ \text{RHS}[\text{NE} \cdot (\text{NE} + 2)]
        \\ \text{RHS}[\text{NE} \cdot (\text{NE} + 2) + 1]
    \end{bmatrix}
\end{equation}



\subsection{The $\omega$ matrix}

The $\omega$ matrix is dedicated to terms that cant be created at the start of the loop and the left as constants. These need to be updated since they are temperature dependant and temperature varies in our iterations.
\vspace{5mm}
\\ For Helmholtz minimization we use $U_j^\circ / \hat{R}T$ which is calculated as $H_j^\circ / \hat{R}T - 1$. We can split many terms inside sums (such as the elemental one) as:

\begin{equation}
    a_{kj} \left( \frac{H_j^\circ}{\hat{R}T} - 1 \right) \mathcal{N}_j = a_{kj} \frac{H_j^\circ}{\hat{R}T} \mathcal{N}_j - a_{kj} \mathcal{N}_j  
\end{equation}

Then, using the linearity of sums, the summation over species becomes:

\begin{equation}
    \sum_{j}^\text{NS} \left[a_{kj} \frac{H_j^\circ}{\hat{R}T} \mathcal{N}_j - a_{kj} \mathcal{N}_j \right]  = \boxed{\sum_{j}^\text{NS} a_{kj} \frac{H_j^\circ}{\hat{R}T} \mathcal{N}_j - \sum_{j=1} ^\text{NS} a_{kj} \mathcal{N}_j}
\end{equation}

This allows us so computational savings since we have already calculated the negative term in the $\xi$ matrix.

\subsubsection{Elemental Rows}

The first NE rows of this matrix looks like:

\begin{equation}
    \begin{bmatrix}
        a_{k1} H_1^\circ \ \hat{R}T & \dots & a_{k, \text{NS}} H_\text{NS}^\circ
        \\ \vdots & \ddots & \vdots
        \\ a_{\text{NE}, 1}, H_1^\circ \ \hat{R}T & \dots & a_{\text{NE}, \text{NS}} H_\text{NS}^\circ
    \end{bmatrix}
\end{equation}

\subsubsection{Charge Rows}

\begin{equation}
    \begin{bmatrix}
        q_1 H_1^\circ / \hat{R}T & \dots & q_\text{NS} H_\text{NS}^\circ / \hat{R}T
    \end{bmatrix}
\end{equation}


\subsubsection{Internal Energy row}

\begin{equation}
    \begin{bmatrix}
        a_{i1} H_1^\circ \ \hat{R}T & \dots & a_{i, \text{NS}} H_\text{NS}^\circ
        \\ \vdots & \ddots & \vdots
        \\ a_{\text{NE}, 1}, H_1^\circ \ \hat{R}T & \dots & a_{\text{NE}, \text{NS}} H_\text{NS}^\circ
        \\ q_1 H_1^\circ / \hat{R}T & \dots & q_\text{NS} H_\text{NS}^\circ / \hat{R}T
        \\ H_1^\circ / \hat{R}T & \dots & H_\text{NS}^\circ / \hat{R}T
        \\ (H_1^\circ / \hat{R}T)^2 & \dots & (H_\text{NS}^\circ / \hat{R}T)^2
        \\ c_{p,1}^\circ / \hat{R} - 1 & \dots & c_{p,\text{NS}}^\circ / \hat{R} - 1
    \end{bmatrix}
\end{equation}

Of course, we note again that the first NE + 1 rows have already been calcualted, so we can reuse their values. Therefore the system is only:


\begin{equation}
    \begin{bmatrix}
        H_1^\circ / \hat{R}T & \dots & H_\text{NS}^\circ / \hat{R}T
        \\ (H_1^\circ / \hat{R}T)^2 & \dots & (H_\text{NS}^\circ / \hat{R}T)^2
        \\ c_{p,1}^\circ / \hat{R} - 1 & \dots & c_{p,\text{NS}}^\circ / \hat{R} - 1
    \end{bmatrix}
\end{equation}


\subsection{RHS}

Many terms that show up on the RHS are calculated from temperature dependant variables.





\section{Coding Tricks}




\end{flushleft}
\end{document}